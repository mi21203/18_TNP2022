\documentclass[aspectratio=169]{beamer}
\usepackage[utf8]{inputenc}
\usepackage{graphics}
\usepackage[english,serbian]{babel}
\usepackage{booktabs} 

\usetheme[sectionstyle=style2]{trigon}
\titlegraphic{\includegraphics[height=\paperheight]{slike/slika1.jpg}} 
\biglogo{LOGO-1.png} 

\title{Žene u računarstvu}
\author{ Zorana Jevtić \and Bojana Zagorac\\  \and Miloš Bigović  \and Jelena Veličković\\}
\date{Decembar 2022}\documentclass[aspectratio=169]{beamer}
\usepackage[utf8]{inputenc}
\usepackage{graphics}
\usepackage[english,serbian]{babel}
\usepackage{booktabs} %za tabelu

\usetheme[sectionstyle=style2]{trigon}
\titlegraphic{\includegraphics[height=\paperheight]{slike/slika1.jpg}} 
\biglogo{LOGO-1.png} 

\title{Žene u računarstvu}
\author{ Zorana Jevtić \and Bojana Zagorac\\  \and Miloš Bigović  \and Jelena Veličković\\}
\date{Decembar 2022}



%==============================================================================
%                               BEGIN DOCUMENT
%==============================================================================


\begin{document}

\titleframe 
\setbeamertemplate{frame footer}{Zorana Jevtić, Bojana Zagorac, Miloš Bigović, Jelena Veličković}

%==============================================

\begin{frame}{Sadržaj}
  \setbeamertemplate{section in toc}[sections numbered]
  \tableofcontents[hideallsubsections]
\end{frame}

%==============================================
\section{Ada Bajron}
%==============================================

\begin{frame}{Biografija}
\begin{itemize}
        \item<1-> Rođena je u Londonu 10. decembra 1815. godine. Otac joj je bio poznati engleski pesnik lord Bajron, a majka Anabela Bajron.

        \item<2-> Svirala je harfu, proučavala poeziju i jahala konje.
        
        \item<3-> Neki naučnici su pokušavali da ospore njen rad, ali je to počelo da se smanjuje kada je čovečanstvo spoznalo moć i važnost programiranja.

        \item<4-> Umrla je od raka sa samo 36 godina, kao i njen otac.

    \end{itemize}
\end{frame}

\begin{frame}{Karijera}
    \begin{itemize}
        \item<1-> U mladosti je pokazivala izvesne matematičke sposobnosti. Imala je lične tutore koji su je podučavali. 
        
        \item<2-> Prilika za napredak i razvoj karijere joj se ukazao upoznavši Čarlsa Bebidža. 
        
        \item<3-> Predosetila je i zagovarala sve ono što se danas smatra modernim programiranjem.
        
        \item<4-> Smatra se prvim programerom na osnovu svog rada i dostignuća.
        
    \end{itemize}
\end{frame}

\begin{frame}
    \begin{columns}
        \column{0.5\textwidth}
        \begin{figure}[h]
    \centering
    \includegraphics[width = .9\textwidth]{adabajron.jpg}
    \caption{Ada Bajron}
    \label{fig:my_label}
\end{figure}

        \column{0.5\textwidth}
        \begin{block}{Dostignuća}
            \begin{itemize}
                \item<1-> Jasno je opisala kako će Bebidžov uređaj funkcionisati svojim prevodom i beleškama koji su objavljeni 1843. godine.

                \item<2-> U njenu čast je jedan softverski jezik nazvan Ada.

                \item<3-> Od 2009. godine dobija priznanje 15. oktobra u cilju podrške i potpore ženama koje se bave naukom.
                
            \end{itemize}
        \end{block}        
    \end{columns}
   
\end{frame}
    

%==============================================
\section{Margaret Hamilton}
%==============================================
\begin{frame}{Biografija}

    \begin{itemize}
        \item<1-> Margaret Hamilton je bila jedan od prvih programera kompjuterskog softvera.
        
        \item<2-> Studirala je matematiku i filozofiju na Earlham koledžu u Ričmondu.
        
        \item<3->Iako je Margaret planirala da studira apstraktnu matematiku, prihvatila je posao na Tehnološkom institutu u Masačusetsu (MIT).
    \end{itemize}


\end{frame}
\begin{frame}{Karijera}

    \begin{itemize}
        \item<1-> Početkom 1960-ih Hamiltonova se pridružila MIT-ovoj Linkoln laboratoriji, gde je bila uključena u projekat Poluautomatsko zemaljsko okruženje, prvi američki sistem protivvazdušne odbrane.
        
        \item<2->Hamilton je zatim radila u kompaniji ''MIT’s Instrumentation Laboratory''. Predvodila je tim koji je imao zadatak da razvije softver za sisteme navođenja i kontrole komandnih i lunarnih modula misija ''Apolo'' u letu.
        
        \item<3-> Sama Margaret se posebno koncentrisala na softver za otkrivanje sistemskih grešaka i za oporavak informacija u slučaju pada računara. Oba ta elementa bila su presudna tokom misije Apolo 11 (1969), koja je odvela astronaute Nila Armstronga i Baza Oldrina na Mesec.
    \end{itemize}


\end{frame}
\begin{frame}
  
    \begin{columns}
        \column{0.5\textwidth}
        \begin{figure}[h]
    \centering
    \includegraphics[width = .4\textwidth]{margaret_hamilton5.jpg}
    \caption{Margaret Hamilton}
    \label{fig:my_label}
\end{figure}

        \column{0.5\textwidth}
        \begin{block}{Dostignuća}
            \begin{itemize}
                \item<1->Bila je jedna od osnivača kompanije Higher Order Software 1976. godine.

                \item<2->Osnovala je Hamilton Technologies.

                \item<3-> Hamilton je bila dobitnica raznih počasti, uključujući NASA-inu nagradu o izuzetnom dostignuću.
                
            \end{itemize}
        \end{block}        
    \end{columns}
   
\end{frame}


%==============================================
\section{ENIAC programeri}
%==============================================

\begin{frame}{Šta je ENIAC}


    \begin{itemize}
        \item<1-> Prvi elektronski računar opšte namene je konstruisan u periodu između 1943. i 1946. godine i dobio je naziv ENIAC.
        
        \item<2-> Osnovna svrha bila mu je jedna specijalna namena - računanje trajektorije projektila.
        
        \item<3-> Mašina je mogla da se preprogramira ali je zahtevalo intervencije na preklopnicima (prekidačima) i kablovima koje su mogle danima da traju. Na programiranju samog računara izdvojile su se šest žena.
    \end{itemize}
    
    \end{frame}
    \begin{frame}{Programeri}
        \begin{list_type}
    
    \item Ketlin Antoneli        
    \item Rut Teitelbaum
    \item Franses Spens          
    \item Žan Bartik
    \item Marlin Melcer          
    \item Beti Holberton

    
    \end{list_type}
    \end{frame}

    \begin{frame}{Kratka biografija}
     
     \begin{itemize}
        \item<1-> Ketlin Antoneli (1921-2006), poreklom iz Irske, je pohađala Čestnat koledž koji je i završila 1942. godine.
        
        \item<2-> Rut Teitelbaum (1924-1986), je završila osnovne akademske studije na Hanter koledžu.
        
        \item<3-> Franses Spens (1922-2012), je upisala Templ Univerzitet. Ubrzo dobija stipendiju Čestnat koledža, gde je upoznala Ketlin Antoneli. 1942. godine nakon završetka studija zaposlila se u vojsci.
        
        \item<4-> Žan Bartik (1924-2011), je upisala Učiteljski koledž sa namerom da studira novinarstvo, međutim ipak se opredelila za informatiku. Zaposlila se na koledžu u Filadelfiji, gde se prvi put susrela sa ENIAC - mašinom.
        
        
    \end{itemize}
        
    \end{frame}

\begin{frame}{Kratka biografija}
    \begin{itemize}
        
        \item<1-> Marlin Melcer (1922-2008), poreklom iz Filadelfije, završila je Templ Univerzitet 1942. godine.
            
        \item<2-> Beti Holberton (1917-2001), je studirala novinarstvo na FIladelfijskom Univerzitetu, što joj je omogućilo da puno proputuje. Međutim, usled rata žene su bile zapošljavane za izračunavanje trajektorije samih projektila.

    \end{itemize}
\end{frame}
    
            

%==============================================
\section{Zastupljenost žena u IT sektoru}
%==============================================

\begin{frame}{Statistika}

    \begin{itemize}
        \item<2-> Posle Drugog svetskog rata, programiranje je postalo popularnije među oba pola.
        
        \item<3->Do sredine 1980-ih godina procenat žena na računarskim naukama rastao je veoma brzo.
        
        \item<4-> Kada su žene počele u ovoj oblasti da čine više od \alert{37\%} studenata, procenat je počeo naglo da pada sve do danas. 
    \end{itemize}

    \begin{table}[h]
        \centering
        \begin{tabular}{c|c}
        \toprule
                Godina    & procenat žena u IT-u \\ 
        \midrule
                1984       & 37\%  \\ 
                1990       & 18\%  \\ 
                2019       & 13\%  \\ 
        \bottomrule
        \end{tabular}
    \end{table}

\end{frame}

%------------------------------------------------

\begin{frame}{Zašto tako malo žena ulazi u IT?}
  
    \begin{columns}
        \column{0.5\textwidth}
            \begin{itemize}
            \item<2-> Žene su manje zainteresovane?
            \item<3-> Kultura, štetni stereotipi
            \item<4-> Nedostatak podrške na školskom nivou 
            \item<5-> Nedostatak ženskih uzora
            \end{itemize}

        \column{0.5\textwidth}
        \begin{figure}[h]
        \centering
        \includegraphics[width = .9\textwidth]{slike/zene_u_ITsektoru.jpg}
        \label{fig:my_label}
        \end{figure}
    \end{columns}
   
\end{frame}

%------------------------------------------------

{\usebackgroundtemplate{\includegraphics[width=\paperwidth, height=\paperheight]{slike/pozadina.jpg}}
\begin{frame}{Budućnost žena u IT-u} 
   
   \begin{itemize}
        \item<2-> Industrija računarskih nauka raste izuzetnom brzinom

        \item<3-> Twitter, Instagram, TikTok su uveli razne heštegove (engl. \textit{hashtag})\\ kojim podržavaju i podstiču rad žena u IT kompanijama  

        \item<4-> Budućnost žena u informatici zavisi od sposobnosti IT\\ industrije da inspiriše mlade žene da se bave Informatikom
   \end{itemize}
   
\end{frame}
}

%=================================================

{\usebackgroundtemplate{\includegraphics[width=\paperwidth, height=\paperheight]{slike/slika_za_kraj.jpg}}
\begin{frame}
    \centering
   \textbf{\Huge{KRAJ}} 
\end{frame}
}

\end{document}




%==============================================================================
%                               BEGIN DOCUMENT
%==============================================================================


\begin{document}

\titleframe 
\setbeamertemplate{frame footer}{Zorana Jevtić, Bojana Zagorac, Miloš Bigović, Jelena Veličković}

%==============================================

\begin{frame}{Sadržaj}
  \setbeamertemplate{section in toc}[sections numbered]
  \tableofcontents[hideallsubsections]
\end{frame}

%==============================================
\section{Ada Bajron}
%==============================================

\begin{frame}{Biografija}
\begin{itemize}
        \item<1-> Rođena je u Londonu 10. decembra 1815. godine. Otac joj je bio poznati engleski pesnik lord Bajron, a majka Anabela Bajron.

        \item<2-> Svirala je harfu, proučavala poeziju i jahala konje.
        
        \item<3-> Neki naučnici su pokušavali da ospore njen rad, ali je to počelo da se smanjuje kada je čovečanstvo spoznalo moć i važnost programiranja.

        \item<4-> Umrla je od raka sa samo 36 godina, kao i njen otac.

    \end{itemize}
\end{frame}

\begin{frame}{Karijera}
    \begin{itemize}
        \item<1-> U mladosti je pokazivala izvesne matematičke sposobnosti. Imala je lične tutore koji su je podučavali. 
        
        \item<2-> Prilika za napredak i razvoj karijere joj se ukazao upoznavši Čarlsa Bebidža. 
        
        \item<3-> Predosetila je i zagovarala sve ono što se danas smatra modernim programiranjem.
        
        \item<4-> Smatra se prvim programerom na osnovu svog rada i dostignuća.
        
    \end{itemize}
\end{frame}

\begin{frame}
    \begin{columns}
        \column{0.5\textwidth}
        \begin{figure}[h]
    \centering
    \includegraphics[width = .9\textwidth]{adabajron.jpg}
    \caption{Ada Bajron}
    \label{fig:my_label}
\end{figure}

        \column{0.5\textwidth}
        \begin{block}{Dostignuća}
            \begin{itemize}
                \item<1-> Jasno je opisala kako će Bebidžov uređaj funkcionisati svojim prevodom i beleškama koji su objavljeni 1843. godine.

                \item<2-> U njenu čast je jedan softverski jezik nazvan Ada.

                \item<3-> Od 2009. godine dobija priznanje 15. oktobra u cilju podrške i potpore ženama koje se bave naukom.
                
            \end{itemize}
        \end{block}        
    \end{columns}
   
\end{frame}
    

%==============================================
\section{Margaret Hamilton}
%==============================================
\begin{frame}{Biografija}

    \begin{itemize}
        \item<1-> Margaret Hamilton je bila jedan od prvih programera kompjuterskog softvera.
        
        \item<2-> Studirala je matematiku i filozofiju na Earlham koledžu u Ričmondu.
        
        \item<3->Iako je Margaret planirala da studira apstraktnu matematiku, prihvatila je posao na Tehnološkom institutu u Masačusetsu (MIT).
    \end{itemize}


\end{frame}
\begin{frame}{Karijera}

    \begin{itemize}
        \item<1-> Početkom 1960-ih Hamiltonova se pridružila MIT-ovoj Linkoln laboratoriji, gde je bila uključena u projekat Poluautomatsko zemaljsko okruženje, prvi američki sistem protivvazdušne odbrane.
        
        \item<2->Hamilton je zatim radila u kompaniji ''MIT’s Instrumentation Laboratory''. Predvodila je tim koji je imao zadatak da razvije softver za sisteme navođenja i kontrole komandnih i lunarnih modula misija ''Apolo'' u letu.
        
        \item<3-> Sama Margaret se posebno koncentrisala na softver za otkrivanje sistemskih grešaka i za oporavak informacija u slučaju pada računara. Oba ta elementa bila su presudna tokom misije Apolo 11 (1969), koja je odvela astronaute Nila Armstronga i Baza Oldrina na Mesec.
    \end{itemize}


\end{frame}
\begin{frame}
  
    \begin{columns}
        \column{0.5\textwidth}
        \begin{figure}[h]
    \centering
    \includegraphics[width = .4\textwidth]{margaret_hamilton5.jpg}
    \caption{Margaret Hamilton}
    \label{fig:my_label}
\end{figure}

        \column{0.5\textwidth}
        \begin{block}{Dostignuća}
            \begin{itemize}
                \item<1->Bila je jedna od osnivača kompanije Higher Order Software 1976. godine.

                \item<2->Osnovala je Hamilton Technologies.

                \item<3-> Hamilton je bila dobitnica raznih počasti, uključujući NASA-inu nagradu o izuzetnom dostignuću.
                
            \end{itemize}
        \end{block}        
    \end{columns}
   
\end{frame}


%==============================================
\section{Eniac programeri}
%==============================================



%==============================================
\section{Zastupljenost žena u IT sektoru}
%==============================================

\begin{frame}{Statistika}

    \begin{itemize}
        \item<2-> Posle Drugog svetskog rata, programiranje je postalo popularnije među oba pola.
        
        \item<3->Do sredine 1980-ih godina procenat žena na računarskim naukama rastao je veoma brzo.
        
        \item<4-> Kada su žene počele u ovoj oblasti da čine više od \alert{37\%} studenata, procenat je počeo naglo da pada sve do danas. 
    \end{itemize}

    \begin{table}[h]
        \centering
        \begin{tabular}{c|c}
        \toprule
                Godina    & procenat žena u IT-u \\ 
        \midrule
                1984       & 37\%  \\ 
                1990       & 18\%  \\ 
                2019       & 13\%  \\ 
        \bottomrule
        \end{tabular}
    \end{table}

\end{frame}

%------------------------------------------------

\begin{frame}{Zašto tako malo žena ulazi u IT?}
  
    \begin{columns}
        \column{0.5\textwidth}
            \begin{itemize}
            \item<2-> Žene su manje zainteresovane?
            \item<3-> Kultura, štetni stereotipi
            \item<4-> Nedostatak podrške na školskom nivou 
            \item<5-> Nedostatak ženskih uzora
            \end{itemize}

        \column{0.5\textwidth}
        \begin{figure}[h]
        \centering
        \includegraphics[width = .9\textwidth]{slike/zene_u_ITsektoru.jpg}
        \label{fig:my_label}
        \end{figure}
    \end{columns}
   
\end{frame}

%------------------------------------------------

{\usebackgroundtemplate{\includegraphics[width=\paperwidth, height=\paperheight]{slike/pozadina.jpg}}
\begin{frame}{Budućnost žena u IT-u} 
   
   \begin{itemize}
        \item<2-> Industrija računarskih nauka raste izuzetnom brzinom

        \item<3-> Twitter, Instagram, TikTok su uveli razne heštegove (engl. \textit{hashtag})\\ kojim podržavaju i podstiču rad žena u IT kompanijama  

        \item<4-> Budućnost žena u informatici zavisi od sposobnosti IT\\ industrije da inspiriše mlade žene da se bave Informatikom
   \end{itemize}
   
\end{frame}
}

%=================================================

{\usebackgroundtemplate{\includegraphics[width=\paperwidth, height=\paperheight]{slike/slika_za_kraj.jpg}}
\begin{frame}
    \centering
   \textbf{\Huge{KRAJ}} 
\end{frame}
}

\end{document}
